\section{Open source distribution, installation}
\label{distr}
The focused crawler has been restructured and packaged as a Debian
package in order to ease distribution and installation. The package
contains dependency information to make sure that all software that is
needed to run the crawler is installed at the same time. In connection
with this we have also packaged a number of necessary Perl-modules as
Debian packages.

\hspace{-\parindent}All software and packages are available from a number of places:
\begin{itemize}
\item the \htmladdnormallinkfoot{Combine focused crawler Web-site}{http://combine.it.lth.se/}
\item the \htmladdnormallinkfoot{Comprehensive Perl Archive Network - CPAN}{http://search.cpan.org/~aardo/Combine/}
\item SourceForge \htmladdnormallinkfoot{project ``Combine focused crawler''}{http://sourceforge.net/projects/focused-crawler}
\end{itemize}

In addition to the distribution sites there is a public
\htmladdnormallinkfoot{discussion list at SourceForge}{http://lists.sourceforge.net/lists/listinfo/focused-crawler-general}.

\subsection{Installation}
This distribution is developed and tested on Linux systems.
It is implemented entirely in Perl and uses the \htmladdnormallinkfoot{MySQL}{http://www.mysql.com/}
database system, both of which are supported on many other
operating systems. Porting to other UNIX dialects should be easy.

The system is distributed either as source or as a Debian package.

\subsubsection{Installation from source for the impatient}
Unless you are on a system supporting Debian packages (in which case look at \hyperref{Automated installation}{Automated installation (section }{)}{debian}), you should
download and unpack the source.
The following command sequence will then install Combine:
\begin{verbatim}
perl Makefile.PL
make
make test
make install
mkdir /etc/combine
cp conf/* /etc/combine/
mkdir /var/run/combine
\end{verbatim}

Test that it all works (run as root)\\
{\tt ./doc/InstallationTest.pl}

\subsubsection{Porting to not supported operating systems - dependencies}
In order to port the system to another platform, you
have to verify the availability, for this platform, of the two main systems:
\begin{itemize}
\item \htmladdnormallinkfoot{Perl}{http://www.cpan.org/ports/index.html}
\item \htmladdnormallinkfoot{MySQL version $\geq$ 4.1}{http://dev.mysql.com/downloads/}
\end{itemize}
If they are supported you stand a good chance to port the system.

Furthermore,
the \hyperref{external Perl modules}{external Perl modules (listed in }{)}{extmods} should be verified to work
on the new platform.

%\begin{verbatim}
%Alvis::Canonical *              HTTP::Date
%Alvis::Pipeline *               HTTP::Status
%Compress::Zlib                  Image::ExifTool
%Config::General                 LWP::UserAgent
%DBI                             Lingua::Identify
%Data::Dumper *                  Lingua::Stem
%Digest::MD5                     MIME::Base64
%Encode                          Net::hostent
%Getopt::Long                    URI
%HTML::Entities                  URI::Escape
%HTML::Tidy *                    URI::URL
%HTML::TokeParser                XML::LibXML *
%\end{verbatim}

Perl modules are most easily installed 
using the Perl CPAN automated system\\
({\tt perl -MCPAN -e shell}).

\hspace{-\parindent}Optionally the following external programs will be used if they are
installed on your system:
\begin{itemize}
 \item antiword (parsing MSWord files)
 \item detex (parsing TeX files)
 \item pdftohtml (parsing PDF files)
 \item pstotext (parsing PS and PDF files, needs ghostview)
 \item xlhtml (parsing MSExcel files)
 \item ppthtml (parsing MSPowerPoint files)
 \item unrtf (parsing RTF files)
 \item tth (parsing TeX files)
 \item untex (parsing TeX files)
\end{itemize}

\subsubsection{Automated Debian/Ubuntu installation}
\label{debian}
\begin{itemize}
\item Add the following lines to your /etc/apt/sources.list:\\
{\tt deb http://combine.it.lth.se/ debian/}
\item Give the commands:\\
{\tt apt-get update\\
apt-get install combine}
\end{itemize}
This also installs all dependencies such as MySQL and a lot of necessary
Perl modules.

\subsubsection{Manual installation}

\htmladdnormallinkfoot{Download the latest distribution}{http://combine.it.lth.se/\#downloads}.

Install all software that Combine depends on (see above).

Unpack the archive with {\tt  tar zxf }\\
This will create a directory named {\tt combine-XX} with
a number of subdirectories including {\tt bin, Combine, doc, and conf}.

'{\tt bin}' contains the executable programs.

'{\tt Combine}' contains needed Perl modules. They should be copied to
where Perl will find them, typically {\tt /usr/share/perl5/Combine/}.

'{\tt conf}' contains the default configuration files. Combine looks for them
in {\tt /etc/combine/} so they need to be copied there.

'{\tt doc}' contains documentation.

The following command sequence will install Combine:
\begin{verbatim}
perl Makefile.PL
make
make test
make install
mkdir /etc/combine
cp conf/* /etc/combine/
mkdir /var/run/combine
\end{verbatim}

\subsubsection{Out-of-the-box installation test}
A simple way to test your newly installed Combine system is
to crawl just one Web-page and export it as an XML-document. This will
exercise much of the code and guarantee that basic focused crawling will work.

\begin{itemize}
\item Initialize a crawl-job named aatest. This will create and populate
the job-specific configuration directory and create the MySQL database
that will hold the records:
\end{itemize}
\begin{verbatim}
sudo combineINIT --jobname aatest --topic /etc/combine/Topic_carnivor.txt 
\end{verbatim}
\begin{itemize}
\item Harvest the test URL by:
\end{itemize}
\begin{verbatim}
combine --jobname aatest
        --harvest http://combine.it.lth.se/CombineTests/InstallationTest.html
\end{verbatim}
\begin{itemize}
\item Export a structured Dublin Core record by:
\end{itemize}
\begin{verbatim}
combineExport --jobname aatest --profile dc
\end{verbatim}
\begin{itemize}
\item and verify that the output, except for dates and order, looks like:
\end{itemize}
\begin{verbatim}
<?xml version="1.0" encoding="UTF-8"?>
<documentCollection version="1.1" xmlns:dc="http://purl.org/dc/elements/1.1/">
<metadata xmlns:dc="http://purl.org/dc/elements/1.1/">
<dc:format>text/html</dc:format>
<dc:format>text/html; charset=iso-8859-1</dc:format>
<dc:subject>Carnivorous plants</dc:subject>
<dc:subject>Drosera</dc:subject>
<dc:subject>Nepenthes</dc:subject>
<dc:title transl="yes">Installation test for Combine</dc:title>
<dc:description></dc:description>
<dc:date>2006-05-19 9:57:03</dc:date>
<dc:identifier>http://combine.it.lth.se/CombineTests/InstallationTest.html</dc:identifier>
<dc:language>en</dc:language>
</metadata>
\end{verbatim}

Or run -- as root -- the script
\hyperref{{\tt ./doc/InstallationTest.pl}}{{\tt ./doc/InstallationTest.pl} (see }{ in the Appendix)}{InstTest}
which essentially does the same thing.


\subsection{Getting started}
\label{gettingstarted}
A simple example work-flow for a trivial crawl job name 'aatest' might look like:

\begin{enumerate}
\item    Initialize database and configuration (needs root privileges)\\
{\tt  sudo combineINIT \verb+--+jobname aatest}
\item \label{load} Load some seed URLs like (you can repeat this command with different URLs as many times as you wish)\\
{\tt  echo 'http://combine.it.lth.se/' | combineCtrl  load \verb+--+jobname aatest}
\item \label{crawl}   Start 2 harvesting processes\\
{\tt  combineCtrl  start \verb+--+jobname aatest \verb+--+harvesters 2}

\item Let it run for some time. Status and progress can be checked using
the program '{\tt combineCtrl \verb+--+jobname aatest}'
with various parameters.

\item    When satisfied kill the crawlers\\
{\tt  combineCtrl kill \verb+--+jobname aatest}
\item    Export data records in the ALVIS XML format\\
{\tt  combineExport \verb+--+jobname aatest \verb+--+profile alvis}

\item If you want to schedule a recheck for all the crawled pages stored in the database do\\
{\tt combineCtrl reharvest \verb+--+jobname aatest}
\item Go back to \ref{crawl} for continuous operation.
\end{enumerate}

Once a job is initialized it is controlled using
{\tt combineCtrl}. Crawled data is exported using {\tt combineExport}.

\subsection{Online documentation}
The latest, updated, detailed documentation is always available
\htmladdnormallinkfoot{online}{http://combine.it.lth.se/documentation/}.

\subsection{Use scenarios}
\subsubsection{General crawling without restrictions}
Use the same procedure as in section \ref{gettingstarted}. This way of
crawling is not recommended for the Combine system since it will
generate really huge databases without any focus.

\subsubsection{Focused crawling -- domain restrictions}
\label{domainfocus}
Create a focused database with all pages from a Web-site. In this
use scenario we will crawl the Combine site and the ALVIS site.
The database is to be continuously updated, i.e. all pages have to be
regularly tested for changes, deleted pages should be removed from
the database, and newly created pages added.
\begin{enumerate}
\item    Initialize database and configuration\\
{\tt  sudo combineINIT \verb+--+jobname focustest}

\item Edit the configuration to provide the desired focus\\
Change the {\tt <allow>} part in {\tt /etc/combine/focustest/combine.cfg} from
\begin{verbatim}
#use either URL or HOST: (obs ':') to match regular expressions to either the
#full URL or the HOST part of a URL.
<allow>
#Allow crawl of URLs or hostnames that matches these regular expressions
HOST: .*$
</allow>
\end{verbatim}
to\\
\begin{verbatim}
#use either URL or HOST: (obs ':') to match regular expressions to either the
#full URL or the HOST part of a URL.
<allow>
#Allow crawl of URLs or hostnames that matches these regular expressions
HOST: www\.alvis\.info$
HOST: combine\.it\.lth\.se$
</allow>
\end{verbatim}
The escaping of '.' by writing '\verb+\.+' is necessary since the patterns
actually are Perl regular expressions. Similarly the ending '\$'
indicates that the host string should end here, so for example
a Web server on {\tt www.alvis.info.com} (if such exists) will
not be crawled.

\item Load seed URLs\\
{\tt  echo 'http://combine.it.lth.se/' | combineCtrl  load \verb+--+jobname~focustest}\\
{\tt  echo 'http://www.alvis.info/' | combineCtrl  load \verb+--+jobname focustest}

\item  Start 1 harvesting process\\
{\tt  combineCtrl  start \verb+--+jobname focustest}

\item  Daily export all data records in the ALVIS XML format\\
{\tt  combineExport \verb+--+jobname focustest \verb+--+profile alvis}\\
and schedule all pages for re-harvesting\\
{\tt combineCtrl reharvest \verb+--+jobname focustest}
\end{enumerate}


\subsubsection{Focused crawling -- topic specific}
\label{topicfocus}
Create and maintain a topic specific crawled database for the topic 'Carnivorous plants'.

\begin{enumerate}
\item Create a topic definition (see section \ref{topicdef}) in a local file named {\tt cpTopic.txt}. (Can be done by copying {\tt /etc/combine/Topic\_carnivor.txt} since it happens to be just that.)

\item Create a file named {\tt cpSeedURLs.txt} with seed URLs for this
topic, containing the URLs:
\begin{verbatim}
http://www.sarracenia.com/faq.html
http://dmoz.org/Home/Gardening/Plants/Carnivorous_Plants/
http://www.omnisterra.com/bot/cp_home.cgi
http://www.vcps.au.com/
http://www.murevarn.se/links.html
\end{verbatim}


\item Initialization\\
{\tt  sudo combineINIT \verb+--+jobname cptest \verb+--+topic cpTopic.txt}

 This enables topic checking and focused crawl mode by setting
configuration variable {\tt doCheckRecord = 1} and copying a topic definition file ({\tt
cpTopic.txt}) to\\ {\tt /etc/combine/cptest/topicdefinition.txt}.

\item Load seed URLs\\
{\tt  combineCtrl  load \verb+--+jobname cptest < cpSeedURLs.txt}

\item  Start 3 harvesting process\\
{\tt  combineCtrl  start \verb+--+jobname cptest \verb+--+harvesters 3}

\item  Regularly export all data records in the ALVIS XML format\\
{\tt  combineExport \verb+--+jobname cptest \verb+--+profile alvis}\\
%and schedule all pages for re-harvesting\\
%{\tt combineCtrl reharvest \verb+--+jobname cptest}

\end{enumerate}

Running this crawler for an extended period will result in more than
200~000 records.

\subsubsection{Focused crawling in an Alvis system}
Use the same procedure as in section \hyperref{Focused crawling -- topic specific}{}{ (Focused crawling -- topic specific)}{topicfocus}
except for the last point. Exporting should be done incrementally into an Alvis
pipeline (in this example listening at port 3333 on the machine nlp.alvis.info):\\
\verb+combineExport --jobname cptest --pipehost nlp.alvis.info --pipeport 3333 --incremental+

\subsubsection{Crawl one entire site and it's outlinks}
This scenario requires the crawler to:
\begin{itemize}
  \item  crawl an entire target site
  \item  crawl all the outlinks from the site
  \item  crawl no other site or URL apart from
         external URLs mentioned on the one target site
\end{itemize}

I.e. all of \verb+http://my.targetsite.com/*+,
plus any other URL that is linked to from a page in
\verb+http://my.targetsite.com/*+.

\begin{enumerate}
\item Configure Combine to crawl this one site only.
Change the {\tt <allow>} part in\\
{\tt /etc/combine/XXX/combine.cfg} to
\begin{verbatim}
#use either URL or HOST: (obs ':') to match regular expressions to either the
#full URL or the HOST part of a URL.
<allow>
#Allow crawl of URLs or hostnames that matches these regular expressions
HOST: my\.targetsite\.com$
</allow>
\end{verbatim}

\item Crawl until you have the entire site (if it's a big site you might want to do the changes
suggested in \hyperref{FAQ no \ref{slowcrawl}}{FAQ no }{}{slowcrawl}).
%FAQ no \ref{slowcrawl}).

\item Stop crawling.

\item Change configuration {\tt <allow>} back to allow crawling
of any domain (which is the default).
\begin{verbatim}
<allow>
#Allow crawl of URLs or hostnames that matches these regular expressions
HOST: .*$
</allow>
\end{verbatim}

\item Schedule all links in the database for crawling, something like (change XXX to your jobname)\\
\verb+echo 'select urlstr from urls;' | mysql -u combine XXX+\\
\verb+          | combineCtrl load --jobname XXX+

\item Change configuration to disable automatic recycling of links:\\
\verb+#Enable(1)/disable(0) automatic recycling of new links+\\
\verb+AutoRecycleLinks = 0+

and maybe (depending or your other requirements) change:\\
\verb+#User agent handles redirects (1) or treat redirects as new links (0)+\\
\verb+UserAgentFollowRedirects = 0+

\item Start crawling and run until no more in queue.
\end{enumerate}
