% % Author:  Jaakko L�fstr�m 10/1/2005
% % mods by Wray

% % note we have pretty much forced document style "article" and pagination details
% % on you, add your EPSF support here or whatever

% % note we have no place for confidentiality considerations, so please
% % place this in bold in your abstract too
% \documentclass[a4paper]{article}
% %\usepackage{alvis}
% \usepackage{listings}
% \usepackage{moreverb}
% \usepackage{listings}

% \title{\YaTeA\\User manual}
% \author{Sophie Aubin, Thierry Hamon}
% \lstset{language=XML}
% \lstset{basicstyle=\footnotesize, }

%    \def\YaTeA{{\rmfamily Y\kern-.36em%
%     \lower.7ex\hbox{A}\kern-.25em%
%     T\kern-.1667em\lower.7ex\hbox{E}\kern-.08emA}~}

% \begin{document}
% \maketitle

\section{User Manual}

\subsection{Introduction}
\label{Introduction}


\YaTeA  aims at identifying and extracting noun phrases
which are potential terms ({\em i.e. } term candidates). Moreover each
term candidate is syntactically analysed in order to identify head and
modifier components. 

Data used for the identification and syntactic analysis of the term
candidates are provided in the DATA directory. They can be modified by
the user. New data can alosi be defined by the user. 

Different kinds of outputs are offered, among which, 
\begin{itemize}
\item an XML annotation of the corpus with either testified terms or
  term candidates in a format compatible with the biolg version of
  the Link Grammar Parser,
\item a list of term candidates in a XML format

\end{itemize}






\subsection{Running \YaTeA}
\label{RunningYatea}


The term extractor can be run according to a command line:

\sloppy
\texttt{perl yatea.pl -lang \textit{language} [-termino
  \textit{file}] [-MNPhtml] [-printChunking] [-compareChunking
  \textit{file}] [-xml\_output] [-suffix] [-correction]
  [-annotate] [-TTforLGP] [-TCforLGP] [termList] [basicTerms]
  file\_in.ttg}


\texttt{file\_in.ttg} is the input corpus in the TreeTagger format.


\textbf{Options} : 
\begin{itemize}
\item \texttt{-lang} \textit{langue}: (mandatory) definition of the
  corpus language (\texttt{EN} for English, \texttt{FR} for
  French). This information leads to select the right configuration files.

\item \texttt{-suffix} : (optional) specification of a name for the current version of the analysis. results are gathered in a specific directory of this name and result files also carry this suffix. 

\item \texttt{-termino} \textit{filename}: (optional) name of a file
  containing testified terms (along with their parse or not)

\item \texttt{-correction}: (optional) allows the correction of POS or
  lemma information of the MNPs according to the testified terms provided.
\end{itemize}
% \item \texttt{-mnp} \textit{type}: (optional) displays \index{display} all the Maximal
%   Noun Phrases (parsed or not) identified in the corpus. \index{The Maximal Noun Phrase} The Maximal Noun Phrases can be displayed with different
%  kinds of information specified by the user in the \texttt{type} field of
%  the option:  \texttt{IF} (inflected form), \texttt{POS} (morpho-syntactic
%  tags), \texttt{LF} (lemmatized form) or \texttt{ALL} (all the
%  types). 

\textbf{Options for result display} : 
\begin{itemize}

\item \texttt{-termList}: (optional) displays a list of terms and
  sub-terms along with their frequency

\item \texttt{-basicTerms}: (optional) displays the list of basic term candidates (every pairs of words in a Head-Modifier relationship)


\item \texttt{-MNPhtml}: (optional) displays \index{display} of the
  corpus marked with MNPs in a HTML file. Information on
  \begin{itemize}
  \item the lexical form of each MNP;
  \item the potential islands of reliability (exogenous and
  indogenous) identified in the MNP;
\item its parsing status (parsed or
  not, parsing method applied)
\item the potential corrections made
  on the lexical information using the testified terms
  \end{itemize}
  can be seen by clicking on each occurrence of an MNP marked in the corpus. 


\item \texttt{-termCandidatesXML}: (optional) displays the analysed term
  candidates in XML format

\item \texttt{-printChunking}: (optional) displays \index{display} of the
  corpus marked with MNPs in a HTML file along with statistical
  information on the corpus lexicon and MNPs. 

\item \texttt{-compareChunking} \textit{filename}: comparison of the chunking (MNP
  identification) of a first version of the current analysed corpus (the option
  -printChunking was specified when analysing the first corpus, here
  specified as \textit{filename}: CORPUS\_Chunking\_corpus.html)

% \item \texttt{-heads}: (optional) selects the display of the terms where the 
%   syntactic head is marked
%   (see subsection \ref{sec:term-head-format}) \index{tete syntaxique\@t�te syntaxique}.

\item \texttt{-annotate} : only annotate testified terms (no acquisition)
\item \texttt{-TTforLGP} : annotation of the corpus with testified terms in a XML format compatible with the "biolg" version of the Link Grammar Parser
\item \texttt{-TCforLGP} : annotation of the corpus with term candidates in a XML format compatible with the "biolg" version of the Link Grammar Parser

\end{itemize}  


\subsubsection{File Hierarchy}

The package provides several directories and files. Configuration
files are use to define specific features for each language. 

\begin{itemize}
\item \verb+yatea.pl+: main program
  % \item \verb+CORPUS+: r�pertoire de corpora (fichiers entr�e) 
  %   \begin{itemize}
  %   \item \verb+LANG+: directory containing the r�pertoire contenant les corpora pour une langue donn�e
  %   \end{itemize}
\item \verb+DATA+: directory containing the linguistic resources
  required for the term extraction and analysis. The resources are
  distributed according to the language.
  \begin{itemize}
  \item \verb+LANG+: directory including the resources for a given
    language (for instance : EN for English) . % We propose to use
    % the acronym defined in the norm ISO 3166 (related to the country code).
    \begin{itemize}
    \item \verb+ChunkingExceptions_LANG+: the file provides the set
      of exceptions items regarding the chunking of the corpus into maximal
      noun phrases. Items can be POS tags or words in their inflected
      or lemmatized form. 
    \item \verb+ChunkingFrontiers_LANG+: the file provides the set
      of items used as frontiers to the chunking of the corpus into maximal
      noun phrases. Items can be POS tags or words in their inflected
      or lemmatized form. 
    \item \verb+CleaningFrontiers_LANG+: the file provides the
      set of items which cannot be used as frontiers of the maximal noun
      phrases. Items can be POS tags or words in their inflected
      or lemmatized form. 
    \item \verb+ForbiddenStructures_LANG+: the file contains
      structures that cannot appear in the noun phrases. Structures can be POS tags or words in their inflected
      or lemmatized form.
    \item \verb+LinguisticOptions_LANG+: options relating to the
      language of the corpus
    \item \verb+MiscellanousOptions_LANG+: other options
    \item \verb+ParsingPatterns_LANG+: the file contains the
      patterns defined to analyse the noun phrases.
    \end{itemize}
  \end{itemize}
  % \item \verb+TOOLS+ : r�pertoire d'outils de segmentation et de nettoyage des textes
\item \verb+RESULTS+ : this directory contains the results of the term
  extraction and analysis.
  \begin{itemize}
  \item \verb+LANG+: this directory is named according to the working
    corpus language provided in command line (\textit{e.g.} \verb+EN+).
    \begin{itemize}
    \item \verb+CORPUS+ or \verb+CORPUS/SUFFIX+: a directory is
      created for each analysed corpus. The directory name is the name
      of the corpus minus its path and extension. If a suffix is
      provided in command line, a sub-directory is created to store the
      results of the current version of the corpus analysis. Results
      are further organized according to the format (\textit{i.e.} extension) of the files:
        \begin{itemize}
        \item \verb+raw+: contains files in text format (or any other)
        \item \verb+html+: contains files in HTML format
        \item \verb+xml+: contains files in XML format
        \end{itemize}
      \end{itemize}
    % Currently, this directory contains the following files (where
%     \verb+<CORPUS>+ is the basename of the corpus without the final
%     extension):
%     \begin{itemize}
%     \item \verb+<CORPUS>_MaximalNounPhrases_TYPE.txt+: list of the maximal
%       noun phrases. TYPE is the type of information requested by the
%       user in the command line option \texttt{-mnp}.

%     \item \verb+<CORPUS>_UnparsedMNP.txt+: list of the noun phrases
%       that could not be parsed.

%     \item \verb+<CORPUS>_ParsedTerms.txt+: list of the parsed
%       noun phrases.

%     \end{itemize}

  \end{itemize}
\item \verb+DOC+ : this directory contains the documentation for the
  term extractor \YaTeA. 
   \begin{itemize}
  %  \item \verb+DeliverableD6.3.pdf+: Description of the architecture
%      and implementation of \YaTeA. 
   \item \verb+Yatea_user_manual.pdf+: this document
     \item \verb+Penn-Treebank-English-Tagset.ps+: description of the
       tagset for English data 
     \item \verb+french-TreeTagger-tagset.html+: description of the
       tagset for French data 
   \end{itemize}
\end{itemize}


\subsection{Input formats}
\label{sec:input-format}

\subsubsection{Corpus}

As an input, the term extractor requires a corpus which has been
segmented in words and sentences, lemmatized and tagged with
part-of-speech (POS)
information. Currently, the prototype takes as input the TreeTagger
ouput format. Information related to a word is recorded in a
line. Each line contains three fields separated by a tabulation:
the inflected form, the POS tag, and the lemma (see sample in 
Figure \ref{corpus}). If the lemma of the word appears as
$<$unknown$>$, its inflected form is used as a lemma.


\begin{figure}
\begin{verbatim}
However         RB      however
,               ,       ,
the             DT      the
concentration   NN      concentration
of              IN      of
protein         NN      protein
p6              NN      <unknown>
appeared        VVD     appear
to              TO      to
be              VB      be
important       JJ      important
for             IN      for
repression      NN      repression
of              IN      of
the             DT      the
early           JJ      early
promoter        NN      promoter
C2              NN      <unknown>
.               SENT    .
\end{verbatim}
  \caption{Sample of the input format}\label{corpus}
\end{figure}

\subsubsection{Resources}\label{ress}

The term extractor requires additional resources (chunking frontiers,
extraction patterns, \emph{etc}.) to process the input corpus. The directory
\texttt{DATA/}\textit{LANG} contains the resources for a given
language, where \textit{LANG} is declared as an option in the
command line. Each file
related to a resource has the suffix \texttt{\_}\textit{LANG}. For
instance, to process an English corpus, the language option is
\texttt{-lang EN}. In that respect, the term extractor uses the resources
in the directory \texttt{DATA/EN}, where each file ends with
\texttt{\_EN} (\texttt{ParsingPatterns\_EN}).

The resources are:


\paragraph{ParsingPatterns}
\label{sec:parsingpatterns}

\texttt{ParsingPatterns\_LANG} contains the definition for the tags
allowed in MNPs and the parsing patterns used to analyse MNPs. It must conform to the following
  format (see sample in Figure \ref{patrons}):
  

  \begin{itemize}
  \item the comments start with the character \verb+#+.
    
  \item four kinds of tags are declared: ``candidate'', ``det'' and 
    ``prep'' which list the tags for respectively content
    words, determiners and  prepositions 
    words. The line starts and ends with the characters
    \texttt{!!}. The tags are defined as regular expressions.
    
    
  \item the syntactic patterns are defined in a line with 3 fields
    separated by a tabulation. The first field describes the syntactic
    analysis of the matching term (\texttt{H} for the head of the
    term, and \texttt{M} for its modifier). The two remaining fields
    give the priority of the pattern (integer) and the analysis
    direction (LEFT or RIGHT).
    The position of the most general head in the pattern determines the direction. 
    
  \end{itemize}
  \begin{figure}[!ht]
\begin{verbatim}
!!candidate = ((JJS)|(JJR)|(JJ)|(NNS)|(NN)|(NP)|(NPS)|(CD)|(VVG))!!
!!det = ((DT)|(PP\$))!!
!!prep = ((of)|(to))!!
!!coo = ((CC))!!

#Pattern                            Priority   Direction
( JJR<=M> NNS<=H> )                     1       RIGHT
( JJ<=M> ( JJ<=M> NNS<=H> )<=H> )       1       RIGHT
( NN<=H> of NNS<=M> )                   1       LEFT
( NN<=H> to DT NN<=M> )                 1       LEFT
\end{verbatim}

    \caption{Sample of the parsing pattern definition file}\label{patrons}
  \end{figure}
  

  \paragraph{ChunkingFrontiers}
  \label{sec:chunkingfrontiers}

  
 \texttt{ChunkingFrontiers\_LANG} contains the declaration of the
 external frontiers of the chunk, \textit{i.e.} tags or words that
 cannot appear in a term (\textit{e.g.} verbs). They are used to chunk the corpus and identify the
  maximal  noun phrases. The required format is given below:

  \begin{itemize}
  \item the comments start with the character \verb+#+.

  \item each line describes a chunking frontier according to two fields:
    the information which will be used as a frontier (inflected form --
    \texttt{IF}, part-of-speech tag -- \texttt{POS}, or lemma --
    \texttt{LF}), and the value of the frontier (see the example in 
    Figure \ref{front_dec}).
    \begin{figure}[!ht]
\begin{verbatim}
# Type of the element   Value of the element
POS     VB
POS     VV
POS     VBD
POS     IN
\end{verbatim}
      \caption{Sample of the chunking frontier file.}\label{front_dec}
    \end{figure}

  \end{itemize}


  \paragraph{ChunkingExceptions}
  \label{sec:chunkingexceptions}

  
 \texttt{ChunkingExceptions\_LANG} contains the declaration of the frontier exceptions.  They are used to prevent the corpus from being 
  chunked in some specific cases. The format is given below:

  \begin{itemize}
  \item the comments start with the character \verb+#+.

  \item each line describes a frontier exception according to two fields:
    the information which will be used as an exception (inflected form --
    \texttt{IF}, part-of-speech tag -- \texttt{POS}, or lemma --
    \texttt{LF}), and the value of the exception (see the example in
    figure \ref{front_dec2} where the inflected form ``of'' is an
    exception of the frontier tag ``IN'' declared in the file of chunking
    frontiers).
    \begin{figure}[!ht]
\begin{verbatim}
# Type of the element   Value of the element
IF     of
\end{verbatim}
      \caption{Sample of the exception frontier file}\label{front_dec2}
    \end{figure}

  \end{itemize}


  \paragraph{CleaningFrontiers}
  \label{sec:cleaningfrontiers}

  
 \texttt{CleaningFrontiers\_LANG} contains the declaration of the cleaning
  frontiers. They are used to improve the chunking
  by removing words (like determiners) that cannot appear in a start
  or final position of a term. 
  % 
  The file contains the list of the elements (words, tags, or lemma) that
  are allowed to start or end a maximal noun phrase. 

The format is given below:

  \begin{itemize}
  \item the comments start with the character \verb+#+.

  \item each line describes a cleaning frontier according to two fields:
    the information which will be used as a frontier (inflected form --
    \texttt{IF}, part-of-speech tag -- \texttt{POS}, or lemma --
    \texttt{LF}), and the value of the frontier (see the example in
    Figure \ref{front_dec3}).
    \begin{figure}[!ht]
\begin{verbatim}
# Type of the element   Value of the element
POS     NN
POS     NNS
POS     NP
\end{verbatim}
      \caption{Sample of the cleaning frontier file}\label{front_dec3}
    \end{figure}

  \end{itemize}


  \paragraph{ForbiddenStructures}
  \label{sec:forbiddenstructures}

   \texttt{ForbiddenStructures\_LANG} contains the declaration of the forbidden structures. They describe the word or
  tag sequences that cannot appear in a term. For instance, in ``\textit{in
    which \textbf{case the timers}}'', the noun phrase ``\textit{case the timers}'' has the
  POS form NN DT NNS and the sequence ``NN DT'' is forbidden. In
  ``\textit{many people}'',  \textit{many} is not allowed to appear in
  a term. 
  There are two possible operations when a forbidden structure is
  found: splitting or deletion, according to what is declared in the file.

The format is given below:

  \begin{itemize}
  \item the comments start with the character \verb+#+.

  \item each line describes a forbidden structure according to five
    fields (the last one is defined only for the splitting action):
     the value of the pattern with the type of
    information (inflected form --
    \texttt{IF}, part-of-speech tag -- \texttt{POS}, or lemma --
    \texttt{LF}), the position in the MNP where the sequences is expected to be
    found (START, END or ANY), the action
    (\texttt{split} or
    \texttt{delete}) for the structures in ANY position, the length in words of the
    pattern, and the word, lemma or tag that will be used to perform
    the splitting (\textit{i.e.} pivot). Figure \ref{front_dec4} shows an example of a forbidden structure file.
    \begin{figure}[!ht]
\begin{verbatim}
# Value\Type info       Position  Action   Length    Pivot
CD\POS of\LF DT\POS     ANY       split	     1	     CD of DT
much\LF datum\LF        ANY       delete
JJ\POS of\LF            START
\end{verbatim}
      \caption{Sample of the forbidden structure file}\label{front_dec4}
    \end{figure}

  \end{itemize}



  \paragraph{LinguisticOptions}
  \label{sec:linguisticoptions}

 \texttt{LinguisticOptions\_LANG} contains the linguistic configuration. The file defines the options for
  the parsing of the terms for a given language. Each option is
  defined on a line as follows:
  \texttt{OPTION\_NAME = value}.

The mandatory options are the following:

\begin{itemize}
\item \texttt{DOCUMENT\_BOUNDARY}: a tag in the corpus that indicates
  the limit between 2 documents
\item \texttt{SENTENCE\_BOUNDARY}: a (POS) tag in the corpus that indicates
  the limit between 2 sentences
\item \texttt{PARSING\_DIRECTION}: the possible values of this option are 
  ``LEFT'' or ``RIGHT''. It defines the tendancy of the given language to
  have the head of a noun phrase on its left or right edge. For
  instance, the preferential direction of the analysis in English is
  ``RIGHT'', while for French it is ``LEFT''. This information is used
  to perform a priority sorting of the parsing patterns when several
  are possible in the prograssive parsing. This information
  refers to the information \textit{Direction} (second field) defined for
  each syntactic pattern in the file \texttt{ParsingPatterns\_LANG}.


\item \texttt{COMPULSORY\_ITEM}: this option is declared
  as a regular expression, using disjonction (OR) and quantifiers
  (\texttt{+}, \texttt{?}, \texttt{*}). It is used during the chunking of
  the noun phrases to check that at least one compulsory tag appears
  in the MNP. For instance, in English, as we use the PennTree Bank tagset with the
  TreeTagger, the option is the following: \texttt{COMPULSORY\_ITEM = (NN.*)}. In that respect, we check that there is
  at least one common noun in each term candidate.
\item \texttt{MNP\_MAXIMUM\_LENGTH} defines the maximal number
  of tokens allowed in a
  maximal noun phrases. The default value is 15.

\end{itemize} 


\paragraph{MiscellaneousOptions}
\label{sec:miscoptions}

 \texttt{MiscellaneousOptions\_LANG} contains the declaration of  miscellaneous options. It defines the processing
  options according to the format \texttt{OptionName = value}:, 

\begin{itemize}
\item \texttt{COLOR\_BLIND} offers brighter colors in the HTML output files 
\item \texttt{TAGGER} defines the path to the tagger used to tag plain
  lists of testified terms
\end{itemize} 

\paragraph{Testified terms}
Lists of testified terms can be provided in different formats:
\textsc{tagged} or \textsc{bracketed}. 

\begin{itemize}
\item 
A \textsc{tagged} foramt terminology is a file containing terms in the TreeTagger
format as shown in
Figure  \ref{TTtagged}. Terms are separated by the SENTENCE\_BOUNDARY tag defined in
the 
LinguisticOptions\_LANG file.
\begin{figure}[!ht]
\begin{verbatim}
catabolite      NN      catabolite
repression      NN      repression
.               SENT    .
core            NN      core
RNA             NN      RNA
polymerase      NN      polymerase
.
\end{verbatim}
      \caption{Sample of a testified term file in \textsc{tagged} format}\label{TTtagged}
    \end{figure}

  \item  
A \textsc{bracketed} format terminology  is a file containing terms along with their
syntactic analysis as shown in
Figure  \ref{TTbracketed}. One term per line is allowed. 6 fields
provide information on each testified: the bracketed parse, the
inflected form, POS sequence and lemmatized form of the term, its
global syntactic head POS tag, and the name of the resource it comes from
\begin{figure}[!ht]
\begin{small}
\begin{verbatim}
( heat<=M=NN> shock<=H=NN> )  heat shock  NN NN  heat shock  NN  PhB
( spore<=M=JJ> coat<=H=NN> )  spore coat  JJ NN  spore coat  NN  PhB
( protein<=H=NN> of ( Bacillus<=H=NN> subtilis<=M=NN> )<=M=NN> )  
  protein of Bacillus subtilis  NN of NN NN  protein of Bacillus 
  subtilis  NN  PhB
\end{verbatim}
\end{small}
      \caption{Sample of a testified term file in \textsc{bracketed} format}\label{TTbracketed}
    \end{figure}

\end{itemize}


\subsection{Output formats}
\label{sec:output-format}

We describe the XML format only in this section. For information on
the other output formats, please see the option description in Section
\ref{RunningYatea}.

\paragraph{XML format}
\label{sec:xmlformat}
The result of the extraction and anlysis of term candidates is
displayed in XML format. The XLM format output allows the display of
non ambiguous term candidates only, \textit{i.e.} term candidates for which
only one syntactic tree was found.

The DTD (Document Type Definition) that was defined for the XML
display appears in Appendix \ref{sec:dtd-xml-display}. Figure \ref{fig:exampleXML} shows an example of the XML output for a
term candidate.


\begin{figure}[!htbp]
  \centering
  {\lstinputlisting[firstline=267,lastline=301]{samples/pmid10788508-v2.genia2.xml}}
  \caption{Example of the XML display for a term candidate}
  \label{fig:exampleXML}
\end{figure}
