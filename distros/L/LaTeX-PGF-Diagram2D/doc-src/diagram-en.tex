\documentclass[origlongtable]{scrartcl}
\KOMAoptions{paper=a4}
\KOMAoptions{fontsize=12pt}
\usepackage{ifpdf}
\usepackage[utf8]{inputenc}
\usepackage[T1]{fontenc}
\usepackage{textcomp}
\usepackage[intlimits]{amsmath}
\usepackage{amssymb}
% \usepackage{amsfonts,amssymb,amscd}
% \usepackage{trfsigns}
\usepackage{mathptmx}
\usepackage[scaled=.92]{helvet}
\usepackage{courier}
\usepackage{array}
\usepackage{enumerate}
\usepackage{longtable}
\usepackage{latexsym}
\usepackage{varioref}
\usepackage{makeidx}
\usepackage{listings}
\lstset{numbers=left, numberstyle=\tiny, numbersep=9pt, basicstyle=\small}
\lstset{frame=single,frameround=tttt}
\usepackage{color}
\ifpdf
\usepackage[activate=normal]{pdfcprot}
\usepackage[pdftex]{graphicx}
% \usepackage{epstopdf}
\pdfcompresslevel=9
\usepackage[
  pdftex,
  a4paper=true,
  colorlinks=true,
  bookmarks,
  bookmarksnumbered,
  linkcolor=black,
  pdfpagemode=UseNone,
  pdfstartview={XYZ 0 835 1.25}
]{hyperref}
\pdfpageattr {/Group << /S /Transparency /I true /CS /DeviceRGB>>}
\else
\usepackage[dvips]{graphicx}
\DeclareGraphicsRule{.png}{eps}{.bb}{`bmeps #1}
\usepackage[dvips]{hyperref}
\fi


\setlength{\parindent}{0cm}


\renewcommand*{\sectfont}{\bfseries}

\newcommand{\startofdocument}[0]{%
\maketitle%
\cleardoublepage%
\ifpdf\pdfbookmark[1]{\contentsname}{tocanc}\fi%
\tableofcontents%
\cleardoublepage%
}

\DeclareMathOperator{\euler}{\text{e}}
\DeclareMathOperator{\oplaplace}{\triangle}
\DeclareMathOperator{\opLaplace}{{\mathfrak L}}
\DeclareMathOperator{\iopLaplace}{{\opLaplace}^{-1}}
\DeclareMathOperator{\definedas}{\overset{=}{\text{\tiny{Def}}}}
\DeclareMathOperator{\definedarrow}{\overset{\longleftrightarrow}{\text{\tiny{Def}}}}
\DeclareMathOperator{\lcorrelates}{\laplace}
\DeclareMathOperator{\ilcorrelates}{\Laplace}
\DeclareMathOperator{\mustbe}{\overset{\text{\tiny !}}{=}}
\DeclareMathOperator{\coj}{\text{j}}
\DeclareMathOperator{\element}{\in}
\DeclareMathOperator{\notelement}{!\in}
\newcommand{\ora}[1]{\overrightarrow{#1}}
\newcommand{\co}[1]{\underline{#1}}
\newcommand{\cove}[1]{\overrightarrow{\underline{#1}}}
\DeclareMathOperator{\grad}{grad}
\DeclareMathOperator{\divergence}{div}
\DeclareMathOperator{\rot}{rot}
\newcommand{\atpos}[2]{{\left.#1\right|}_{#2}}
\newcommand{\inrange}[3]{{\left.#1\right|}_{#2}^{#3}}
\newcommand{\intd}[0]{\mathrm{d}}
\DeclareMathOperator{\arcsinh}{arcsinh}
\DeclareMathOperator{\card}{card}
\DeclareMathOperator{\Res}{Res}
\DeclareMathOperator{\expe}{e}

\newcommand{\begriff}[2]{%
  \emph{#1}\index{#2}%
}

\usepackage{attachfile}

\makeindex




% Image base file name
% Caption
\newcommand{\figimage}[2]{%
\begin{figure}%
{\centering%
\input{#1.tex}%
\caption{#2}%
\label{fig:#1}%
}%
\end{figure}%
}

% Linewidth factor
% Image base file name
% Caption
\newcommand{\figimagelw}[3]{%
\begin{figure}%
{\centering%
\resizebox{#1\linewidth}{!}{\input{#2.tex}}%
\caption{#3}%
\label{fig:#1}%
}%
\end{figure}%
}

% Width in cm
% Image base file name
% Caption
\newcommand{\figimagecm}[3]{}

% Image base file name
% Caption
\newcommand{\pngimage}[2]{%
\begin{figure}%
{\centering%
\ifpdf%
\input{#1.pdf}%
\else%
\input{#1}%
\fi%
\caption{#2}%
\label{fig:#1}%
}%
\end{figure}%
}

% Linewidth factor
% Image base file name
% Caption
\newcommand{\pngimagelw}[3]{%
\begin{figure}%
{\centering%
\ifpdf%
\includegraphics[width=#1\linewidth]{#2.pdf}%
\else%
\includegraphics[width=#1\linewidth]{#2}%
\fi%
\caption{#3}%
\label{fig:#1}%
}%
\end{figure}%
}

% Width in cm
% Image base file name
% Caption
\newcommand{\pngimagecm}[3]{%
\begin{figure}%
{\centering%
\ifpdf%
\includegraphics[width=#1cm]{#2.pdf}%
\else%
\includegraphics[width=#1cm]{#2}%
\fi%
\caption{#3}%
\label{fig:#1}%
}%
\end{figure}%
}

% Image base file name
% Caption
\newcommand{\mpimage}[2]{%
\begin{figure}%
{\centering%
\includegraphics{#1.mps}%
\caption{#2}%
\label{fig:#1}%
}%
\end{figure}%
}

% Linewidth factor
% Image base file name
% Caption
\newcommand{\mpimagelw}[3]{%
\begin{figure}%
{\centering%
\includegraphics[width=#1\linewidth]{#2.mps}%
\caption{#3}%
\label{fig:#1}%
}%
\end{figure}%
}

% Width in cm
% Image base file name
% Caption
\newcommand{\mpimagecm}[3]{%
\begin{figure}%
{\centering%
\includegraphics[width=#1cm]{#2.mps}%
\caption{#3}%
\label{fig:#1}%
}%
\end{figure}%
}


\newcommand{\gpimage}[2]{\mpimage{#1}{#2}}
\newcommand{\gpimagelw}[3]{\mpimagelw{#1}{#2}{#3}}
\newcommand{\gpimagecm}[3]{\mpimagecm{#1}{#2}{#3}}


\hyphenation{
Ar-beits-grup-pe
Win-dow-Event
Ar-beits-platz
Komp-ri-mier-ung
komp-ri-mier-te
Aus-gangs-ver-zeich-nis
Standard-ein-gabe
Dia-gramm
Dia-gram-Rand
Ska-len-teil-ung
}

\usepackage{fancybox}
\usepackage{pgf}
% \usepackage{ngerman}
\newcommand{\version}[0]{1.00}
\author{Dipl.-Ing.~D.~Krause}
\title{\resizebox{10cm}{!}{LaTeX::PGF::Diagram2D}}
\ifpdf
\hypersetup{
  pdftitle={LaTeX::PGF::Diagram2D 1.00},
  pdfsubject={LaTeX::PGF::Diagram2D},
  pdfauthor={Dipl.-Ing.~D.~Krause}
}
\fi

\newcommand{\includepgfimage}[2]{%
\begin{figure}%
{\centering%
\input{#1.pgf}\caption{#2}\label{fig:#1}%
}%
\end{figure}%
}
\begin{document}
\startofdocument
\section{Introduction}
The module plots 2D diagrams following the \textsc{din} 461 standard
for use with \LaTeX{}.

The diagrams can contain functions and/or data.

PGF files produced by the module can be used with both latex/dvips
and pdflatex.

Use code like
\begin{lstlisting}
\begin{figure}%
{\centering%
\input{xxx.pgf}\caption{Meine Bildunterschrift}\label{blabla}%
}%
\end{figure}%
\end{lstlisting}
to use a PGF file in your \LaTeX{} source.

In contrast to other plotting programs we specify the size of the
drawing grid, not the size of the entire image.

Bezier splines are used to draw the curve unless polylines are
required explicitly.

The module does not attempt to replace other mathematics and
plotting software, instead it is best used in combination
with
GnuPlot\footnote{\url{http://gnuplot.sourceforge.net}}
and
wxMaxima\footnote{\url{http://maxima.sourceforge.net}}.

\clearpage
\section{Tutorial}
\subsection[Function]{Example 1: Function plot}
\figimage{diagf003}{Example circuit}

In an electronic circuit as seen in figure~\vref{fig:diagf003}
we have a voltage source \(U_{\text{q}}=1\,\text{V}\) with an internal resistance
\(R_{i}=4\,\Omega\). For a variable load resistor
\(0\leq R_{\text{L}}\leq 10\,\Omega\) we want to print the current
\(I\) for each \(R_{\text{L}}\) value.

The formula
\begin{align*}I&=\frac{U_{\text{q}}}{R_{\text{L}}+R_{\text{i}}}\end{align*}
can be used to calculate \(I\).

For a first overview we use GnuPlot:
\begin{lstlisting}[language=GnuPlot]
Uq=1.0
Ri=4.0
set xrange [0:10.0]
plot Uq/(Ri+x) title "Strom"
\end{lstlisting}
The maximum current is \(\tfrac{1}{4}\,\text{A}\), so we use a grid with
of 10\,cm and a grid height of 6\,cm to show currents up to 0,3\,A.

\clearpage
\lstinputlisting[float,title={../examples/test001a.pl},language=Perl]{../examples/test001a.pl}
For the first plot we use the file ../examples/test001a.pl.

In line 3 we use the LaTeX::PGF::Diagram2D module.

The function \textit{I} calculates the current for a resistor value,
the resistor value is the function argument (line 8\ldots{}13).

``LaTeX::PGF::Diagram2D->new'' creates a new diagram, the grid size
is \(10\times{}6\)\,cm (line 16).

The image is used in a \LaTeX{} source written for a font size of
12\.pt (line 18).

The lower \(x\) axis (b: bottom) covers a range 0,0\ldots{}10,0
using a linear scale. Grid lines are drawn for a distance of 1 (line 20),
grid values are printed for a distance of 1 (line 21).

The left \(y\) axis (l: left) covers a range 0,0\ldots{}0,3.
We draw grid lines for a distance of 0,05 and create grid values
for a distance of 0,1 (lines 22 and 23).

The \textit{axis()\/} method of the diagram object returns a reference
for an axis object. A number of axis object methods is used to
set up the axis. Each of the methods returns the axis object reference,
so the methods can be cascaded.

The \textit{plot()\/} method of the diagram object creates a new
plot object and returns the reference to it (line 23).
The \textit{set\textunderscore{}xy\textunderscore{}fct()\/} method
configures the plot object to use the \textit{I()\/} to calculate
the values (line 26).

We save the diagram to file ``../examples/test001a.pgf'' (line 28).

\includepgfimage{../examples/test001a}{Current depending on resistor value.}
\clearpage
\subsection[Labels]{Example 2: Labels and units}
\lstinputlisting[float,language=Perl,title={../examples/test002a.pl}]{../examples/test002a.pl}
\includepgfimage{../examples/test002a}{Diagram with labels and units.}
The \textit{set\textunderscore{}label()\/} axis method sets the
axis label.
The \textit{set\textunderscore{}unit()\/} method sets the unit.
The label is normally set in math mode, the unit is written upright.
Both label and unit are passed to \LaTeX{} so we can use \LaTeX{}
markup here. For each backslash we want to pass to \LaTeX{} we have
to write two backslashes in the Perl source code.

The arrows are added automatically to the axis labels.
The unit is placed between the last two scale values.
\clearpage
\subsection[Space for unit]{Example 3: More space for the unit}
\includepgfimage{../examples/test003a}{More space for \(x\) axis unit}
In the previous example there is very few space between ``9'', ``\(\Omega\)''
and ``10''. So we decide to omit one scale value. If necessary we could
omit multiple values but we are not allowed to omit value ``0''.

The \textit{set\textunderscore{}omit()\/} is used to specify the number
of values to omit.
\lstinputlisting[float,language=Perl,title={../examples/test003a.pl}]{../examples/test003a.pl}
\clearpage
\subsection[Multiple curves]{Example 4: Multiple curves}
\includepgfimage{../examples/test004a}{Multiple curves in one diagram}
We can draw multiple curves into one diagram, we simply have to
create and configure multiple plot objects.
In the example we create a second plot for an internal resistance
of \(6\,\Omega\).

Before we change a variable which is a parameter of a plot function
we must call the \textit{finish()\/} method for all the previous
plot objects. This method does all the calculations needed for the
plot, it can be called after completely configuring the plot.
If the method is not called explicitly, it is called automatically
when the output file is written.

In the example we configure the plot object for the first curve
using \(R_{\text{i}}=4\,\Omega\). Before we change the
``\textdollar{}Ri'' variable to 6.0 we have to calls the
\textit{finish()\/} method of this plot object.
\lstinputlisting[float,language=Perl,title={../examples/test004a.pl}]{../examples/test004a.pl}
\clearpage
\subsection[Text labels]{Example 5: Text labels}
\includepgfimage{../examples/test005a}{Text labels}
Text labels can be used to identify curves. The
\textit{label()\/} diagram method creates a text label.
The arguments are the axis names, the coordinates and the label text.
Optionally you can specify a text anchor.

\lstinputlisting[float,language=Perl,title={../examples/test005a.pl}]{../examples/test005a.pl}
\clearpage
\subsection[Two scales]{Example 6: Two scales}
\includepgfimage{../examples/test006a}{Two scales}
Now we want to print the current and the power dissipation of the
load resistor for \(R_{\text{i}}=4\,\Omega\).
The power dissipation of the load resistor can be calculated as:
\begin{align*}P&=\frac{{U_{\text{q}}}^2R_{\text{L}}}{{(R_{\text{L}}+R_{\text{i}})}^2}\end{align*}
We use GnuPlot for a first overview:
\begin{lstlisting}[language=GnuPlot]
Uq=1.0
Ri=4.0
set xrange [0:10.0]
plot Uq*Uq*x/((Ri+x)*(Ri+x)) title "power dissipation"
\end{lstlisting}
We have a maximum \(P_{\text{max}}=0,0625\,\text{W}\) for
\(R_{\text{L}}=4\,\Omega\).
So we increase the diagram height to 8\,cm. Now the left \(y\) axis
covers a current range 0,0\ldots{}0,4\,A, the right axis covers a
power range 0,0\ldots{}0,08\,W.

For the right axis we have to configure some distances manually:
\begin{itemize}
\item	\textit{set\textunderscore{}tic\textunderscore{}offset()\/}\\
Distance between right grid border and right edge of scale values.
The default 2\,mm is useful for the left axis, for the right axis
we must correct the value.
\item	\textit{set\textunderscore{}label\textunderscore{}offset()\/}\\
Distance between right grid border and axis label center.
\item	\textit{set\textunderscore{}border()\/}\\
Distance between right grid border and right image border.
\end{itemize}
\lstinputlisting[float,language=Perl,title={../examples/test006a.pl}]{../examples/test006a.pl}
\clearpage
\subsection[Units]{Example 7: Units}
\includepgfimage{../examples/test007a}{Changed units}
To improve the diagram we use mA and mW as units. We have to change
the calculation functions (Multiplication of the result by factor 1000)
and the axis settings.

The coordinates for the text labels must be changed too.
\lstinputlisting[float,language=Perl,title={../examples/test007a.pl}]{../examples/test007a.pl}
\clearpage
\subsection[Derivative]{Example 8: Specify 1st derivative}
\includepgfimage{../examples/test008a}{First derivative}
The module creates Bezier splines for curve drawing, to calculate
the control points we need the first derivative for the calculated
curve points.
If no function to calculate the the first derivative is specified
the module attempts a natural spline interpolation and calculates
the derivative for the interpolation. For \(N\) points we have
\(N-1\) segments, so we have to solve a system of
\(4(N-1)\) linear equations. This can take ``some time'', so it should
be avoided.

We can use wxMaxima to calculate \(\frac{\intd I}{\intd R}\)
and \(\frac{\intd P}{\intd R}\):
\begin{lstlisting}
diff((1000*Uq/(Ri+RL)), RL);
ratsimp(diff((1000*Uq*Uq*RL/((Ri+RL)*(Ri+RL))), RL));
\end{lstlisting}
For
\begin{align*}
I(R_{\text{L}})&=\frac{1000\cdot{}U_{\text{q}}}{R_{\text{i}}+R_{\text{L}}}&P(R_{\text{L}})&=\frac{1000\cdot{}U_{\text{q}}R_{\text{L}}^2}{{(R_{\text{i}}+R_{\text{L}})}^2}\\
\intertext{we find}
\frac{\intd I}{\intd R_{\text{L}}}&=-\frac{1000\cdot{}U_{\text{q}}}{{(R_{\text{i}}+R_{\text{L}})}^2}&\frac{\intd P}{\intd R_{\text{L}}}&=-\frac{1000\cdot{} U_{\text{q}}^2(R_{\text{L}}-R_{\text{i}})}{R_{\text{}L}^3+3R_{\text{i}}R_{\text{L}}^2+3R_{\text{i}}^2R_{\text{L}}+3R_{\text{i}}^3}\end{align*}

The reference to the function for the derivative calculation is passed
as the second (optional) argument to
\textit{set\textunderscore{}xy\textunderscore{}fct()\/}
(lines 39 and 40).
\lstinputlisting[float,language=Perl,title={../examples/test008a.pl}]{../examples/test008a.pl}
\clearpage
\subsection[Parametric plot]{Example 9: Parametric plot}
\includepgfimage{../examples/test009a}{Parametric plot}
In the example we draw the curve:
\begin{align*}
\tilde{k}:&&x&=5\expe^{-t/2\pi}\cos{t}&y&=5\expe^{-t/2\pi}\sin{t}&0&\leq{}t\leq{}6\pi
\end{align*}

For the functions
\begin{align*}
x(t)&=5\cdot\expe^{-t/2\pi}\cos{t}&y(t)&=5\cdot\expe^{-t/2\pi}\sin{t}\\
\intertext{we find the first derivative:}
\frac{\intd x}{\intd t}&=-\frac{1}{2\pi}(10\pi\sin{t}+5\cos{t})\cdot\expe^{-t/2\pi}&\frac{\intd y}{\intd t}&=-\frac{1}{2\pi}(5\sin{t}-10\pi\cos{t})\expe^{-t/2\pi}\end{align*}
For a given \(t\) the functions \textit{xfct()\/} and \textit{yfct()\/}
can be used to calculate \(x\) and \(y\).
\textit{dxdt()\/} and \textit{dydt()\/} are used to calculate
\(\tfrac{\intd x}{\intd t}\) and \(\tfrac{\intd y}{\intd t}\).

The start and end value of \(t\) and the function references to
\textit{xfct()\/} and \textit{yfct()\/} are passed as arguments
to \textit{set\textunderscore{}parametric\textunderscore{}fct()\/}.
The two optional arguments are references to functions calculating
the first derivative. It is recommended to specify these arguments.

\lstinputlisting[float,language=Perl,title={../examples/test009a.pl}]{../examples/test009a.pl}
\clearpage
\subsection[Polyline]{Example 10: Curve or polylinie}
\includepgfimage{../examples/test011a}{Polyline}
Bezier splines are suitable for ``smooth'' functions as in be examples
before. For ``unsmooth'' curves (i.\,e. rectangle\ldots) we use
a polyline instead, the \textit{set\textunderscore{}lines()\/}
method of the plot object can be used to configure this.
Additionally we use \textit{set\textunderscore{}intervals()\/} to increase
the number of segments.
\clearpage
\lstinputlisting[float,language=Perl,title={../examples/test011a.pl}]{../examples/test011a.pl}
\clearpage
\subsection[Logarithmic scale]{Example 11: Logarithmic scale}
\includepgfimage{../examples/test012a}{Low-pass transfer function}
In the example we show the low-pass voltage transfer
function:
\begin{align*}
\frac{v_U}{\text{dB}}&=20\cdot\lg{\left|\frac{1}{1+\coj\frac{f}{f_{\text{G}}}}\right|}\\[0.2em]
&=-10\lg{\left({\left(\frac{f}{f_{\text{G}}}\right)}^2+1\right)}\end{align*}

A logarithmic scale is used for the \(x\) axis.
The substitution
\begin{align*}
X&=\frac{f}{f_{\text{G}}}\\
\intertext{results in}
Y&=-10\lg{\left(X^2+1\right)}\\[0.2em]
\frac{\intd Y}{\intd X}&=-\frac{20X}{\left(X^2+1\right)\ln{10}}
\end{align*}
The
\textit{set\textunderscore{}logarithmic()\/} axis object method
sets up a logarithmic scale, both start and end value must be positive.
The values used in
\textit{set\textunderscore{}tic\textunderscore{}step()\/} and
\textit{set\textunderscore{}grid\textunderscore{}step()\/}
are increase/decrease factors instead of deltas.

In the example we draw grid lines for \(x\) values increasing by
factor 10.
\clearpage
\lstinputlisting[float,language=Perl,title={../examples/test012a.pl}]{../examples/test012a.pl}
\clearpage
\subsection[Data]{Example 12: Plot data}
In the example we have measurement results \(U_{C2}=f(U{B1})\)
for a two-stage transistor amplifier (voltage of stage 2 collector
depending on voltage of stage 1 base).\\[1em]
\begin{tabular}{ccc}
\begin{minipage}[m]{0.45\linewidth}
\begin{tabular}{|c|c|}
\hline
\(U_{B1}\) / V&\(U_{C2}\) / V\\
\hline
\hline
&\\[-0.9em]0,350&~0,17\\
\hline
&\\[-0.9em]0,355&~0,21\\
\hline
&\\[-0.9em]0,360&~0,26\\
\hline
&\\[-0.9em]0,365&~0,31\\
\hline
&\\[-0.9em]0,370&~0,38\\
\hline
&\\[-0.9em]0,375&~0,45\\
\hline
&\\[-0.9em]0,380&~0,54\\
\hline
&\\[-0.9em]0,385&~0,65\\
\hline
&\\[-0.9em]0,390&~0,79\\
\hline
&\\[-0.9em]0,395&~0,93\\
\hline
&\\[-0.9em]0,400&~1,10\\
\hline
&\\[-0.9em]0,405&~1,33\\
\hline
&\\[-0.9em]0,410&~1,62\\
\hline
&\\[-0.9em]0,415&~1,90\\
\hline
&\\[-0.9em]0,420&~2,27\\
\hline
&\\[-0.9em]0,425&~2,66\\
\hline
\end{tabular}
\end{minipage}&~\begin{minipage}[m]{0.45\linewidth}
\begin{tabular}{|c|c|}
\hline
\(U_{B1}\) / V&\(U_{C2}\) / V\\
\hline
\hline
&\\[-0.9em]0,430&~3,16\\
\hline
&\\[-0.9em]0,435&~3,65\\
\hline
&\\[-0.9em]0,440&~4,35\\
\hline
&\\[-0.9em]0,445&~5,04\\
\hline
&\\[-0.9em]0,450&~5,86\\
\hline
&\\[-0.9em]0,455&~6,64\\
\hline
&\\[-0.9em]0,460&~7,58\\
\hline
&\\[-0.9em]0,465&~8,69\\
\hline
&\\[-0.9em]0,470&~9,92\\
\hline
&\\[-0.9em]0,475&11,20\\
\hline
&\\[-0.9em]0,480&12,53\\
\hline
&\\[-0.9em]0,485&13,74\\
\hline
&\\[-0.9em]0,490&15,01\\
\hline
&\\[-0.9em]0,495&15,07\\
\hline
&\\[-0.9em]0,500&15,09\\
\hline
&\\[-0.9em]&\\
\hline
\end{tabular}
\end{minipage}
\end{tabular}\\[1em]
The \textit{set\textunderscore{}xy\textunderscore{}points\textunderscore{}text()\/}
plot object method uses a text variable containing the measurement
points, one per line. The first value in the line is the \(X\) value,
the second value is the \(Y\) value. Optionally a line can contain the
derivative value as the third argument as seen for  \(U_{B1}=0,49\,\text{V}\).
This value is only used for natural spline interpolations.

The
\textit{set\textunderscore{}xy\textunderscore{}points\textunderscore{}file()\/}
plot object method reads the point data from file.

The
\textit{set\textunderscore{}xy\textunderscore{}points()\/} plot object method
uses a reference to an array. The array itself contains references to
two-or-three element arrays containing the point data.

The
\textit{set\textunderscore{}dots()\/} plot object method switches from
curve printing to dots printing. Without this instruction a natural
spline interpolation is printed.

\clearpage
\includepgfimage{../examples/test013a}{Measurement results}
\clearpage
\lstinputlisting[float,language=Perl,title={../examples/test013a.pl}]{../examples/test013a.pl}
\clearpage
\includepgfimage{../examples/test014a}{Measurement results curve}
\clearpage
\lstinputlisting[float,language=Perl,title={../examples/test014a.pl}]{../examples/test014a.pl}
\clearpage
\subsection[X-splines]{Example 13: Data plot using X-splines}
\includepgfimage{../examples/test015a}{Data plot using X-splines}
The LaTeX::PGF::Diagram2D modules curve drawing doesn't smooth the
curve, so the previous examples looks a little ``rough''.
To smooth the curve we can use X-splines.
Cross-splines (X-splines) were introduced by Carole Blanc und Christophe Schlick
on the \textsc{siggraph}'95. In the conference proceedings you can find the
article online as:
``X-Splines: A Spline Model Designed for the End-User''.\\
An X-spline is a contiguous
curve specified by a set of control points. For each control point
we have the coordinates \(x\) and \(y\) and a paramter \(s\).
For \(-1\leq{}s<0\) the curve is interpolated in the point.
For \(0\leq{}x\leq{}1\) the curve is approximated in the point.
For \(s=0\) a bend is allowed in the point. Positive values allow
the curve to go away from the point. Negative values towards
-1 require interpolation with small curvature.

In the example we require a curve interpolation with small
curvature radius for \(U_{\text{B1}}=0,490\,\text{V}\) and
approximation for all other points.

\clearpage
\lstinputlisting[float,language=Perl,title={../examples/test015a.pl}]{../examples/test015a.pl}
\clearpage
\section{Reference}
\subsection[Diagram]{LaTeX::PGF::Diagram2D}
\subsubsection{Overview}
Each LaTeX::PGF::Diagram2D object represents a diagram.
\subsubsection{Constructor}
\begin{itemize}
\item	\textit{LaTeX::PGF::Diagram2D-\textgreater{}new(width, height)\/}\\
creates a new diagram. Width and height of the grid area are specified
in cm.
\end{itemize}
\subsubsection{Setup}
\begin{itemize}
\item	\textit{set\textunderscore{}font\textunderscore{}size(size)\/}\\
specifies the the font size in point (pt). This is used to calculate
the scale value \(y\) position for the lower \(x\) axis.
\item	\textit{axis(name)\/}\\
returns a reference to a LaTeX::PGF::Diagram2D::Axis object which can
be used to set up the axis. Use ``bottom'', ``left'', ``right'' and ``top''
or the abbreviations ``b'', ``l'', ``r'' or ``t'' in the name.
\end{itemize}
\subsubsection{Plot objects}
\begin{itemize}
\item	\textit{plot(name,name)\/}\\
creates a new LaTeX::PGF::Diagram2D::Plot object and returns the object
reference. The reference can be used to configure the plot.
The arguments are the names of the \(x\) and \(y\) axis to use for the plot.
\item	\textit{copy\textunderscore{}plot(plot)\/}\\
creates a copy of an existing LaTeX::PGF::Diagram2D::Plot object and
returns the reference to the new object.
\end{itemize}
\subsubsection{Additional graphics objects}
\begin{itemize}
\item	\textit{label(name, name, x, y, text [, position])\/}\\
creates a new text object. The first two arguments are the names of
\(x\) and \(y\) axis to use, followed by \(x\) and \(y\) position
and the text. The optional position parameter specifies the anchor
of the text (``left'', ``right'', ``top'', ``bottom'' or a combination
like ``right,bottom''). The default is to center the label both
horizontally and vertically.
\item	\textit{polyline(name, name, arrayreference)\/}\\
creates a polyline object. You have to name the \(x\) and \(y\) axis,
the third argument is a reference to an array containing \(x\) and
\(y\) value for each point.
\end{itemize}
\subsubsection{Output}
\begin{itemize}
\item	\textit{write(filename)\/}\\
creates the named file containing the diagram. If the file name
suffix is ``.tex'' a full \LaTeX{} source for processing with pdflatex
is created.
By default the program creates a PGF file which can be referenced
from a *.tex file.
\end{itemize}
\clearpage
\subsection[Axis]{LaTeX::PGF::Diagram2D::Axis}
\subsubsection{Overview}
A LaTeX::PGF::Diagram2D::Axis object represents one of the four diagram
(grid) borders:
\begin{itemize}
\item	b (bottom \(x\)-axis),
\item	l (left \(y\)-axis),
\item	r (right \(y\)-axis),
\item	t (top \(x\)-axis).
\end{itemize}
The objects are created automatically by the LaTeX::PGF::Diagram2D
constructor. You can obtain object references by calling a
LaTeX::PGF::Diagram2D objects \textit{get\textunderscore{}axis(name)\/}
method, the name can be one from ``bottom'' (``b''), ``left'' (``l''),
``right'' (``r'') or ``top'' (``t'').

\subsubsection{Scale range}
\begin{itemize}
\item	\textit{set\textunderscore{}linear(min, max)\/}\\
sets up linear scaling for the specified range.
\item	\textit{set\textunderscore{}logarithmic(min, max)}\\
sets up logarithmic scaling for the specified range.
Both min and max must be positive.
\end{itemize}

\subsubsection{Grid setup}
\begin{itemize}
\item	\textit{set\textunderscore{}grid\textunderscore{}step(step)\/}\\
specifies the grid distance (only bottom and left axis).
The step value is a constant distance for linear scales and an
increase/decrease factor for logarithmic scales.
\item	\textit{set\textunderscore{}tic\textunderscore{}step(step)\/}\\
specified the distance for scale values.
\end{itemize}

\subsubsection{Axis labels and units}
\begin{itemize}
\item	\textit{set\textunderscore{}label(text)\/}\\
sets the text for the axis label (typically written in math mode).
\item	\textit{set\textunderscore{}unit(text)\/}\\
sets the text for the unit (typically written upright).
\item	\textit{set\textunderscore{}omit(number)\/}\\
sets the number of scale values to omit (default: 0) to typeset
the unit.
\item	\textit{set\textunderscore{}color(color)\/}\\
sets the color for axis label and arrow. Use \LaTeX{} color specifications
like ``blue'' or ``blue!50!black''.
Only the color of axis label and arrow is changed, the scale values
remain black.
\end{itemize}
\clearpage
\subsubsection{Regions and distances}
\figimagelw{}{diagf002}{Abstände}
For each axis ``to'' (tic offset) specifies the distance from grid border to
right, upper or lower edge of scale values.
``lo'' (label offset) is the distance from grid border to the center
of the axis label.
``bo'' (border) is the distance between grid border and image border.
\begin{itemize}
\item	\textit{set\textunderscore{}tic\textunderscore{}offset(distance)\/}\hfill{}sets to.
\item	\textit{set\textunderscore{}label\textunderscore{}offset(distance)\/}\hfill{}sets lo.
\item	\textit{set\textunderscore{}border(distance)\/}\hfill{}sets bo.
\end{itemize}
\clearpage
\subsection[Plot]{LaTeX::PGF::Diagram2D::Plot}
\subsubsection{Overview}
A LaTeX::PGF::Diagram2D::Plot object represents data to plot in a diagram,
either a point set or a curve.
\subsubsection{Constructor}
Objects are constructed in the \textit{plot()\/} or
\textit{copy\textunderscore{}plot()\/} method of a 
LaTeX::PGF::Diagram2D object.
\subsubsection{Data source}
\begin{itemize}
\item	\textit{set\textunderscore{}xy\textunderscore{}fct(fct [, derivative ])}\/\\
uses the specified function(s) to calculate plot points.
The \textit{fct()\/} function is used to calculate \(y\) values.
The use of the optional second parameter -- a function to calculate the
derivative -- is recommended.
\item	\textit{set\textunderscore{}parametric\textunderscore{}fct(min, max, xfct, yfct [, xderivative, yderivative ])\/}\\
turns on parametric plotting. The \(t\) parameter is in the range
min\ldots{}max, xfct and yfct are references to functions to calculate
\(x(t)\) and \(y(t)\). The use of the optional xderivative and yderivative
parameters -- functions to calculate \(\frac{\intd x}{\intd t}\)
and \(\frac{\intd y}{\intd t}\) -- is recommended.
\item	\textit{set\textunderscore{}xy\textunderscore{}points(arrayref)\/}\\
uses points from an array specified by reference. The array itself contains
references to two- or three-element array containing \(x\) and \(y\)
values and optionally the first derivative in the point.
\item	\textit{set\textunderscore{}xy\textunderscore{}points\textunderscore{}text(text)\/}\\
uses a text containing point data, one point per line.
For each point we have \(x\) and \(y\) value and optionally the
derivative in the point.
\item	\textit{set\textunderscore{}xy\textunderscore{}points\textunderscore{}file(filename)\/}\\
uses a text file containing the point data.
\item	\textit{set\textunderscore{}xsplines\textunderscore{}points(arrayref)}\\
uses an array of point data to draw using an X-spline.
The array contains references to 2- or 3-element array
containing \(x\), \(y\) and optionally \(s\) for each point.
The default \(s\) value is -1 for interpolation.
\item	\textit{set\textunderscore{}xsplines\textunderscore{}points\textunderscore{}text(text)}\\
uses a text variable to produce an X-spline.
\item	\textit{set\textunderscore{}xsplines\textunderscore{}points\textunderscore{}file(filename)\/}\\
uses the specified file to produce an X-spline.
\item	\textit{set\textunderscore{}xsplines\textunderscore{}segments(number)}\\
sets the number of Bezier spline segments to draw for each X-spline
segment. Normally the default value 8 is sufficient.
\end{itemize}
\subsubsection{Configure output}
\begin{itemize}
\item	\textit{set\textunderscore{}intervals( number )\/}\\
sets the number of intervals (Bezier spline segments) for function plotting.
\item	\textit{set\textunderscore{}curve()\/}\\
sets drawing mode to ``curve'' (default).
\item	\textit{set\textunderscore{}lines()\/}\\
sets drawing mode to ``polyline''.
\item	\textit{set\textunderscore{}dots([ style [, size ]  ])\/}\\
sets drawing mode to ``point set''.
The optional
\textit{style} parameter specifies how to draw the point:
"`circle"', "`square"', "`diamond"', "`triangle"', "`crosshair"' or
"`pentagon"'.\\
The second optional parameter specifies the size of the point
relative to line width.
\item	\textit{set\textunderscore{}color(name)\/}\\
sets the plot color, use \LaTeX{} color specifications here.
\end{itemize}
\subsubsection{Preparing print}
\begin{itemize}
\item	\textit{finish()\/}\\
prepares the plot object for printing. All coordinates (curve points and
control points) needed for drawing are calculated now. This function is
called implicitly by the diagrams \textit{write()\/} method if necessary.
An explicit call to this method is necessary before changing
parameters of the functions used.
\end{itemize}
\clearpage
\subsection{Label}
Each ``LaTeX::PGF::Diagram2D::Label'' object represents one additional
text label printed into the diagram.
The LaTeX::PGF::Diagram2D object's \textit{label()\/} method creates a new
label object and returns the object reference.

The object's \textit{set\_color()\/} method can be used to set the label
color, the argument must be a \LaTeX{} color specification.
\clearpage
\subsection{Polyline}
A ``LaTeX::PGF::Diagram2D::Polyline'' object represents an additional
polyline drawn into the diagram.
The LaTeX::PGF::Diagram2D object's \textit{polyline()\/} method creates a new
polyline object and returns the object reference.

The object's \textit{set\_color()\/} method can be used to set the label
color, the argument must be a \LaTeX{} color specification.

The line width can be set using the \textit{set\_width()\/} method, the
argument is a factor. The line width is calculated as
\(\text{factor}\cdot{}\text{curve line width}\).
\clearpage
\section{Hints}
\begin{itemize}
\item	\textbf{Specify the first derivative if known}\\
When plotting functions as curves the first derivative
\begin{align*}\frac{\intd Y}{\intd X}&\end{align*} is needed for each curve
point, for parametric plots
\begin{align*}\frac{\intd X}{\intd T}&&\frac{\intd Y}{\intd T}\end{align*}
to calculate the corresponding derivative values in paper coordinates
\begin{align*}\frac{\intd y}{\intd x}&&\frac{\intd x}{\intd t}&&\frac{\intd y}{\intd t}\end{align*}
If the derivative values are not known the module attempts a natural spline
interpolation through the curve points and calculates the derivatives from
the natural spline coefficients.\\
For each spline segment there are 4 coefficients, for \(u\) segments we
have \(4u\) coefficients, this requires to solve a system of \(4u\)
linear equations. For a high number of segments this might take
``some time''.\\
Drawing Bezier segments with ``real'' derivatives will probably result
in better curves than using ``guessed'' derivatives.
\item	\textbf{Use black on white.}\\
\textsc{din} 461 doesn't mention any color at all. Sometimes people need to
reproduce technical documents ``in the field''. You should make sure your
diagrams can be copied  on a black/white copy or on the next fax by
using black or at least dark colors. Additionally when plotting
multiple curves into one diagram make sure curves can be identified
by adding text labels.
\end{itemize}
\clearpage
\section{Theory}
\subsection{Coordinates}
2D plots show \(Y\) depending on \(X\), on paper we use the
coordinates \(x\) and \(y\).
For parametric plots \(X\) and \(Y\) depend on parameter \(T\) running
through a specified interval.

For the axis we specify the value range
\([X_{\text{min}};X_{\text{max}}]\) which is represented by the
coordinate range
\([x_{\text{min}};x_{\text{max}}]\) on paper.

Linear or logarithmic scales can be used.

For linear scales we can calculate \(x\):
\begin{align*}
x&=x_{\text{min}}+\frac{X-X_{\text{min}}}{X_{\text{max}}-X_{\text{min}}}\cdot{}(x_{\text{max}}-x_{\text{min}})\\
\intertext{and \(\frac{\intd x}{\intd X}\):}
\frac{\intd x}{\intd X}&=\frac{x_{\text{max}}-x_{\text{min}}}{X_{\text{max}}-X_{\text{min}}}\end{align*}
For logarithmic scales we calculate \(x\):
\begin{align*}
x&=x_{\text{min}}+(x_{\text{max}}-x_{\text{min}})\frac{\log{\left(\frac{X}{X_{\text{min}}}\right)}}{\log{\left(\frac{X_{\text{max}}}{X_{\text{min}}}\right)}}\\
\intertext{and \(\frac{\intd x}{\intd X}\):}
\frac{\intd x}{\intd X}&=\frac{x_{\text{max}}-x_{\text{min}}}{X\log{\left(\frac{X_{\text{max}}}{X_{\text{min}}}\right)}}
\end{align*}
For logarithmic scales \(\frac{\intd x}{\intd X}\) depends on \(X\), for
linear scales the value is constant.

For the \(y\) axis the calculations are similar.

These equations are sufficient to draw polylines and dot sets.
\clearpage
\subsection{Plots using Bezier splines}
\subsubsection{Bezier segments}
For function plotting the module calculates a number of points.
The points are equidistant on paper (same \(\Delta x\)). For logarithmic
scales this means the \(X\) values differ by a constant factor.

The curve consists of several segments. Modern graphics file formats
(i.\,e. PostScript, PDF and PGF used here) use Bezier segements. Each Bezier
segment is specified by four points:
start point \(P_0(x_0;y_0)\), first control point \(P_0^+(x_0^+;y_0^+)\),
second control point \(P_1^-(x_1^-;y_1^-)\) and end point \(P_1(x_1;y_1)\).

The segment is a parametric curve:
\begin{align*}
0&\leq{}t\leq 1\\[0.2em]
x(t)&={(1-t)}^3x_0+3t{(1-t)}^2x_0^++3t^2(1-t)x_1^-+t^3x_1\\[0.2em]
y(t)&={(1-t)}^3y_0+3t{(1-t)}^2y_0^++3t^2(1-t)y_1^-+t^3y_1
\end{align*}
We calculate the first derivatives for \(t=0\) and \(t=1\):
\begin{align*}
\atpos{\frac{\intd x}{\intd t}}{t=0}&=3x_0^+-3x_0&\atpos{\frac{\intd x}{\intd t}}{t=1}&=3x_1-3x_1^-\\[0.2em]
\atpos{\frac{\intd y}{\intd t}}{t=0}&=3y_0^+-3y_0&\atpos{\frac{\intd y}{\intd t}}{t=1}&=3y_1-3y_1^-\\
\intertext{If we know \(P_0\) and \(P_1\) and the derivatives \(\tfrac{\intd x}{\intd t}\) and \(\tfrac{\intd y}{\intd t}\) we can calculate the control points:}
x_0^+&=x_0+\frac{1}{3}\cdot\atpos{\frac{\intd x}{\intd t}}{t=0}&x_1^-&=x_1-\frac{1}{3}\cdot\atpos{\frac{\intd x}{\intd t}}{t=1}\\[0.2em]
y_0^+&=y_0+\frac{1}{3}\cdot\atpos{\frac{\intd y}{\intd t}}{t=0}&y_1^-&=y_1-\frac{1}{3}\cdot\atpos{\frac{\intd y}{\intd t}}{t=1}
\end{align*}
\clearpage
\subsubsection{Plots y=f(x)}
For plots \(y=f(x)\) we can do a substitution for each segment with given
\[
x_0\qquad{}y_0\qquad\atpos{\frac{\intd y}{\intd x}}{x=x_0}\qquad{}x_1\qquad{}y_1\qquad\atpos{\frac{\intd y}{\intd x}}{x=x_1}
\]
to:
\begin{align*}
x(t)&=x_0+(x_1-x_0)\cdot t&t\in[0;1]\\[0.2em]
\frac{\intd x}{\intd t}&=x_1-x_0&\frac{\intd y}{\intd t}&=\frac{\intd y}{\intd x}\cdot\frac{\intd x}{\intd t}
\end{align*}
If we know \(\tfrac{\intd Y}{\intd X}\) we can calculate:
\begin{align*}
\frac{\intd y}{\intd x}&=\frac{\frac{\intd y}{\intd Y}}{\frac{\intd x}{\intd X}}\cdot\frac{\intd Y}{\intd X}
\end{align*}
If we don't know \(\tfrac{\intd Y}{\intd X}\) we have to do a curve
interpolation, see section~\vref{sec:interpolation}.
\clearpage
\subsubsection{Parametric plots}
For curve segments \(T_0\)\ldots\(T_1\) we substitute:
\begin{align*}
T(t)&=T_0+(T_1-T_0)\cdot t&t&\in[0;1]\\[0.2em]
\frac{\intd T}{\intd t}&=T_1-T_0\end{align*}
If we know \(\tfrac{\intd X}{\intd T}\) and
\(\tfrac{\intd Y}{\intd T}\) we can calculate
\(\tfrac{\intd x}{\intd t}\)
and \(\tfrac{\intd y}{\intd t}\):
\begin{align*}
\frac{\intd x}{\intd t}&=\frac{\intd x}{\intd X}\cdot\frac{\intd X}{\intd T}\cdot\frac{\intd T}{\intd t}\\[0.2em]
\frac{\intd y}{\intd t}&=\frac{\intd y}{\intd Y}\cdot\frac{\intd Y}{\intd T}\cdot\frac{\intd T}{\intd t}
\end{align*}
Otherwise we have to do a curve interpolation.
\clearpage
\subsection{Curve interpolation}\label{sec:interpolation}
\subsubsection{Overview}
Curve interpolation means to search for a function or a sequence of
functions running through specified points.

We can either use one function or a composite function.

An example for one function is a high-degree polynomial.
For \(N\) points we have a system of \(N\) linear equations giving
the \(N\) coefficients for a \({N-1}^{\text{th}}\) degree polynomial.
The drawback of such a polynomial is that there are \(N-2\) extremes.
This might result in an oscillating curve, so I don't know of any
practical use of polynomial interpolation.

A composite function (sequence of functions, one per interval) can
produce smoother curves with less oscillations.
Two variants of composite functions are natural splines and X-splines.

\subsubsection{Natural Splines}
For \(N\) specified points with strict monotonous growing
\(x\)- or. \(t\)-values
we have \(N-1\) segments. Each segment \(i\) between
\(x\)-values \(x_i\) and \(x_{i+1}\)
is described by a cubic polynomial
\begin{align*}
y&=m_i{(x-x_i)}^3+n{(x-x_i)}^2+p_i(x-x_i)+q_i\\
\intertext{respectively}
y&=a_ix^3+b_ix^2+c_ix+d_i
\end{align*}
Both notations are equivalent and can be converted forward and backward.
We use the second notation here.
\clearpage
For \(N-1\) curve segements we need to find \(4(N-1)\) coefficients,
\(4(N-1)\) equations are necessary:
\begin{itemize}
\item	N-1 equations (0\ldots{}N-2) for the value at the left border
of each segment (\(0\leq i\leq N-1\)):
\begin{align*}y_i&=a_ix_i^3+b_ix_i^2+c_ix_i+d_i\end{align*}
\item	N-1 equations (N-1\ldots{}2N-3) for the value at the right border
of each segment (\(0\leq i\leq N-1\)):
\begin{align*}y_{i+1}&=a_ix_{i+1}^3+b_ix_{i+1}^2+c_ix_{i+1}+d_i\end{align*}
\item	N-2 equations (2N-2\ldots{}3N-5) for inner points, having the
first derivative from the left and right segment equal (\(0\leq i\leq N-2\)):
\begin{align*}3a_ix_{i+1}^2+2b_ix_{i+1}+c_i&=3a_{i+1}x_{i+1}^2+2b_{i+1}x_{i+1}+c_{i+1}\end{align*}
\item	N-2 equations (3N-4\ldots{}4N-7) for inner points, having the
second derivative from the left and right segment equal (\(0\leq i\leq N-2\)):
\begin{align*}6a_ix_{i+1}+2b_i&=6a_{i+1}x_{i+1}+2b_{i+1}\end{align*}
\item	2 equations (4N-6 and 4N-5) to specify that the second derivative
on the outer points is 0:
\begin{align*}
6a_0x_0+2b_0&=0\\[0.2em]
6a_{N-2}x_{N-1}+2b_{N-2}&=0\end{align*}
\end{itemize}
The solution of the system of linear equations is a set of
(\(a_i\), \(b_i\), \(c_i\) and \(d_i\)).

If values are given for the first derivative, we have to substitute
equations:
For inner points we replace equation
\(3N-4+i-1\),
(second derivative of left and right segment equal) by
\begin{align*}3a_ix_i^2+2b_ix_i+c_i&=y'(x_i)\end{align*}
For the start point we replace \(4N-6\)
(second derivative = 0)
by:
\begin{align*}3a_0x_0^2+2b_0x_0+c_0&=y'(x_0)\end{align*}
For the last point we replace \(4N-5\) (second derivative = 0)
by:
\begin{align*}3a_{N-2}x_{N-1}^2+2b_{N-2}x_{N-1}+c_{N-2}&=y'(x_{N-1})\end{align*}

The ``PDL'' module is used to solve the system of linear equations.
\clearpage
\subsubsection{X-Splines}
Cross-splines (X-splines) were introduced by Carole Blanc und Christophe Schlick
on the \textsc{siggraph}'95. In the conference proceedings you can find the
article online as:
``X-Splines: A Spline Model Designed for the End-User''.\\
In this documentation I'll explain X-splines only minimally, please
read the \textsc{siggraph}'95 paper for a full description.

An X-spline is a set of discrete control points describing a contiguous
curve. For each control point \(i\) there are values
\(x_i\), \(y_i\) and \(s_i\) (for 3D-graphics not covered here there
might be a \(z_i\)).
A point can be an approximation point (\(s_i>0\)), an interpolation
point (\(s_i<0\)) or a bend (\(s_i=0\)). \(s_i\) values going towards
1 allow the curve to go away from the control point.
\(s_i\) values going towards -1 require large radii for the interpolation
curve in the control point.
\clearpage
\subsection{Examples}
\subsubsection{Point set}
\includepgfimage{../examples/test016a}{Point set}
In this section we show some examples for curve approximation and
interpolation using the point set shown here.
\clearpage
\subsubsection{Higher-degree polynomial}
First we create an \(8^{\text{th}}\) degree polynomial through the points.
\begin{align*}
y&=ax^8+bx^7+cx^6+dx^5+ex^4+fx^3+gx^2+hx+i
\end{align*}
The numeric solution of
\begin{align*}
\left(\begin{array}{ccccccccc}
1^8&1^7&1^6&1^5&1^4&1^3&1^2&1&1\\
2^8&2^7&2^6&2^5&2^4&2^3&2^2&2&1\\
3^8&3^7&3^6&3^5&3^4&3^3&3^2&3&1\\
4^8&4^7&4^6&4^5&4^4&4^3&4^2&4&1\\
5^8&5^7&5^6&5^5&5^4&5^3&5^2&5&1\\
6^8&6^7&6^6&6^5&6^4&6^3&6^2&6&1\\
7^8&7^7&7^6&7^5&7^4&7^3&7^2&7&1\\
8^8&8^7&8^6&8^5&8^4&8^3&8^2&8&1\\
9^8&9^7&9^6&9^5&9^4&9^3&9^2&9&1\\
\end{array}\right)\cdot\left(\begin{array}{c}a\\b\\c\\d\\e\\f\\g\\h\\i\end{array}\right)&=\left(\begin{array}{c}2\\2\\2\\2\\4\\6\\6\\6\\6\end{array}\right)
\end{align*}
results in:
\begin{align*}
a&=1,7647\cdot{}{10}^{-16}&b&=-1,9841\cdot{}{10}^{-3}&c&=6,9444\cdot{}{10}^{-2}&d&=-9,8056\cdot{}{10}^{-1}\\[0.2em]
e&=7,1528&f&=-28,722&g&=62,778&h&=-68,295\\[0.2em]
i&=30
\end{align*}
Figure~\vref{fig:../examples/test017a}
shows the polyonmial. We see oscillations.
\clearpage
\includepgfimage{../examples/test017a}{\(8^{\text{th}}\) degree polynomial}
\clearpage
\subsubsection{Natural spline}
Now we use a natural spline interpolcation.
\lstinputlisting[language=Perl,title={../examples/test018a.pl}]{../examples/test018a.pl}

There are still some smaller oscilations, see figure~\vref{fig:../examples/test018a}.
\clearpage
\includepgfimage{../examples/test018a}{Natural spline}
\clearpage
For the points (4;2) and (6;6) we specify the derivative to be 0,
as the curve ``comes from the left'' and ``goes to the right''.
\lstinputlisting[language=Perl,title={../examples/test019a.pl}]{../examples/test019a.pl}
Result (figure~\vref{fig:../examples/test019a}): No oscillations anymore.
\clearpage
\includepgfimage{../examples/test019a}{Natural spline, derivative specified}
\clearpage
\subsubsection{X-Splines}
Now we do an interpolation using X-splines. We do not specify an
\(s\) value, so the default -1 is used for interpolation.
\lstinputlisting[language=Perl,title={../examples/test020a.pl}]{../examples/test020a.pl}
See figure~\vref{fig:../examples/test020a} for the result.
\clearpage
\includepgfimage{../examples/test020a}{X-spline interpolation}
\clearpage
Now we choose a smaller \(s\)-value for (4;2) and (6;6)
(\(-\tfrac{1}{2}\)). This allows smaller radii in these points
and reduces oscillations.
\lstinputlisting[language=Perl,title={../examples/test021a.pl}]{../examples/test021a.pl}
\includepgfimage{../examples/test021a}{X-spline interpolation, smaller curve radii}
\clearpage
For approximation we use
\(s=1,0\) we specify the value as second argument to
\textit{set\textunderscore{}xsplines\textunderscore{}points\textunderscore{}text()\/}.
The default value is used for all lines not containing an \(s\) value.
\lstinputlisting[language=Perl,title={../examples/test022a.pl}]{../examples/test022a.pl}
Figure~\vref{fig:../examples/test022a} shows the result.
\clearpage
\includepgfimage{../examples/test022a}{X-spline approximation}
\clearpage
For (4;2) and (6;6) we change \(s=0,5\) to drag the curve towards these points.
\lstinputlisting[language=Perl,title={../examples/test023a.pl}]{../examples/test023a.pl}
\clearpage
\includepgfimage{../examples/test023a}{X-spline approxmiation dragging the curve towards points}
\clearpage
In the final curve we use different \(s\) values to show that approximation
and interpolation can be used in different ranges of the same curve.
\lstinputlisting[language=Perl,title={../examples/test024a.pl}]{../examples/test024a.pl}
\clearpage
\includepgfimage{../examples/test024a}{X-spline approximation and interpolation}
\clearpage
\end{document}

